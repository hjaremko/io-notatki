\documentclass[../main.tex]{subfiles}


\begin{document}

    \begin{table}[H]
        \begin{center}
            \begin{tabular}{  p{6cm} p{10cm}  }
                \textbf{Inżynieria oprogramowania}
                &
                Zastosowanie \textbf{inżynierskiego} (systematycznego, zdyscyplinowanego, ilościowego) \textbf{podejścia}
                do oprogramowania (rozwoju, eksploatacji, utrzymania).
                \\
                \cmidrule(r){1-1}\cmidrule(l){2-2}
                Produkcja oprogramowania
                &
                \begin{itemize}
                    \item analiza,
                    \item wymagania,
                    \item projektowanie,
                    \item wdrożenie, ewolucja systemu
                \end{itemize}
                \\
                \cmidrule(r){1-1}\cmidrule(l){2-2}
                Proces tworzenia oprogramowania
                &
                \textbf{Zbiór czynności} i związanych z nimi wyników, \textbf{prowadzących do powstania
                systemu} informatycznego - tworzenie oprogramowania od zera, rozszerzanie i modyfikowanie istniejących systemów.
                \\

                \cmidrule(r){1-1}\cmidrule(l){2-2}

                \textbf{Produkt} - wewnętrzny lub docelowy
                &
                \begin{itemize}
                    \item specyfikacja
                    \item podręcznik użytkowania
                    \item scenariusze przypadków użycia
                    \item raport o statusie projektu
                    \item podręcznik testera
                \end{itemize}
                \\
                \cmidrule(r){1-1}\cmidrule(l){2-2}
                \textbf{Aktywności}, zadania i \textbf{zasoby}.
                &
                \begin{itemize}
                    \item zbieranie, analizowanie wymagań
                    \item realizacja przypadku użycia
                    \item projektowanie systemu, obiektów
                    \item implementacja, testowanie
                    \item komunikacja,
                    \item zarządzanie konfiguracją, projektem
                    \item cykl życiowy oprogramowania
                \end{itemize}
                \\
            \end{tabular}
        \end{center}
    \end{table}
    \begin{table}[H]
        \begin{center}
            \begin{tabular}{  p{6cm} p{10cm}  }

                \textbf{Scenariusz przypadku użycia} - wyspecyfikowana \underline{sekwencja zdarzeń} między użytkownikiem a systemem.
                &
                \begin{itemize}
                    \item Zdefiniowane w pierwszej kolejności.
                    \item Wyróżnia się jeden \textbf{główny scenariusz sukcesu}.
                    \item Może zawierać warunki wstępne, gwarancje lub wyzwalacze.
                    \item W agile development używa się \underline{skróconej wersji scenariusza} odpowiadającej
                    na pytania:kto, co, dlaczego.
                \end{itemize}
                \\

                \cmidrule(r){1-1}\cmidrule(l){2-2}

                \textbf{Przypadek użycia} - \underline{zbiór powiązanych ze sobą scenariuszy} opisujących użycie systemu przez aktorów.
                &
                    Opisujemy je tekstowo, poprzez user stories lub diagramy.
                    \begin{itemize}
                        \item reprezentuje \textbf{funkcjonalne wymaganie} systemu;
                        \item pewna historia; opisuje akcje systemu z punktu widzenia użytkownika;
                        \item specyfikuje jeden aspekt zachowania bez wchodzenia w strukturę systemu;
                        \item jest zorientowany na osiągnięcie celu użytkownika;
                    \end{itemize}
                \\

                \end{tabular}
        \end{center}
    \end{table}






\end{document}