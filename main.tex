\documentclass[a4paper]{article}

\usepackage{fullpage} % Package to use full page
\usepackage{parskip} % Package to tweak paragraph skipping
\usepackage{tikz} % Package for drawing
\usepackage{amsmath}
\usepackage{hyperref}
\usepackage[utf8]{inputenc}
\usepackage{graphicx}
\usepackage{enumitem}
\usepackage{booktabs}
\usepackage{lmodern}
\usepackage[MeX]{polski}
\usepackage[T1]{fontenc}
\usepackage{float}
\usepackage{subfiles}
\usepackage{pdfpages}



\title{Notatki z kursu Inżynieria Oprogramowania}
\author{Małgorzata Dymek}
\date{2018/19, semestr letni}

\graphicspath{{graphics/}}

\begin{document}
    \maketitle

    \section{Podstawowe pojęcia}
    \subfile{sections/podstawowe_pojecia}

    \section {UML}
    \subfile{sections/uml}


    \section{Procesy wytwarzania oprogramowania}
    \subfile{sections/procesy}


    \section{Standardy jakości}
    \subfile{sections/standardy}

    \section{Zwinne procesy wytwarzania oprogramowania}
    \subfile{sections/zwinne}


    \section{Wymagania}
    \subfile{sections/wymagania}

    \section{Projektowanie systemu}
    \subfile{sections/system}

    \section{Projektowanie obiektów}
    \subfile{sections/obiekty}

    \section{Testowanie i kontrola jakości}
    \subfile{sections/test}


    \section{Ewolucja oprogramowania i zarządzanie konfiguracją}
    \subfile{sections/ewolucja}

    \section{Ciągła integracja, oprogramowanie w chmurze}
    \subfile{sections/integracja}



\end{document}

