\documentclass[../main.tex]{subfiles}

\begin{document}

    \subsection{Przykre zapachy w kodzie programów}

    \begin{table}[H]
        \begin{center}
            \begin{tabular}{ p{.35\linewidth} p{.65\linewidth} }
                \toprule
                \textbf{Nazwa} & \textbf{Rozwiązanie}\\
                \toprule
                \textbf{Zduplikowany kod}
                \begin{itemize}
                    \item Identyczny lub podobny kod w wielu miejscach
                \end{itemize}
                &
                \begin{itemize}
                    \item w jednej klasie: \textit{Extract Method}
                    \item w klasach o wspólnej nadklasie: \textit{Extract Method} i \textit{Pull-up Method}
                    \item w klasach niezwiązanych: \textit{Extract Class}
                \end{itemize}
                \\

                \cmidrule(r){1-1}\cmidrule(l){2-2}
                \textbf{Długa metoda}

                \begin{itemize}
                    \item Metoda wykonuje w rzeczywistości wiele czynności
                    \item Brak wsparcia ze strony metod niższego poziomu
                \end{itemize}
                &
                \begin{itemize}
                    \item \textit{Extract Method}
                    \item \textit{Replace Temp with Query}
                    \item \textit{Replace Method with Method Object}
                    \item \textit{Introduce Parameter Object} lub \textit{Preserve Whole Object}
                \end{itemize}
                \\

                \cmidrule(r){1-1}\cmidrule(l){2-2}
                \textbf{Nadmiernie rozbudowana klasa}

                \begin{itemize}
                    \item Klasa posiada zbyt wiele odpowiedzialności,
                    \item Liczne klasy wewnętrzne, pola i metody,
                    \item Duża liczba metod upraszczających.
                \end{itemize}
                &
                \begin{itemize}
                    \item \textit{Extract Class}, \textit{Extract Sub/Superclass} lub \textit{Extract Interface}
                    \item \textit{Pull up Member}, \textit{Push Down Member}, \textit{Move Member}.
                \end{itemize}
                \\

                \cmidrule(r){1-1}\cmidrule(l){2-2}
                \textbf{Długa lista parametrów}
                \begin{itemize}
                    \item Metoda otrzymuje więcej danych niż potrzebuje
                \end{itemize}
                &
                \begin{itemize}
                    \item \textit{Replace Parameter with Method}, \textit{Preserve Whole Object} lub \textit{Introduce Parameter Object}.
                \end{itemize}
                \\

                \cmidrule(r){1-1}\cmidrule(l){2-2}
                \textbf{ Nadmiar komentarzy}
                \begin{itemize}
                    \item Komentarze niepotrzebnie opisują znaczenie kodu
                \end{itemize}
                &
                \begin{itemize}
                    \item \textit{Extract Metod}, \textit{Rename Metod}
                    \item \textit{Introduce Assertion}
                \end{itemize}
                \\

                \cmidrule(r){1-1}\cmidrule(l){2-2}
                \textbf{Niekompletna klasa biblioteczna}
                \begin{itemize}
                    \item Gotowa biblioteka nie posiada pewnej funkcjonalności
                \end{itemize}
                &
                \begin{itemize}
                    \item \textit{Introduce Foreign Method}, \textit{Introduce Local Extension}
                \end{itemize}
                \\

                \cmidrule(r){1-1}\cmidrule(l){2-2}
                \textbf{Skomplikowane instrukcje warunkowe}
                \begin{itemize}
                    \item Metoda zawiera złożoną instrukcję if lub switch
                \end{itemize}
                &
                \begin{itemize}
                    \item \textit{Extract Method}
                    \item \textit{Replace Conditional with Polimorphism/State}
                    \item \textit{Replace Conditional with Subclasses}
                \end{itemize}
                \\

                \cmidrule(r){1-1}\cmidrule(l){2-2}
                \textbf{Łańcuchy wywołań metod}
                \begin{itemize}
                    \item łańcuhc wywołań przez delegację
                    \item naruszenie prawa Demeter
                \end{itemize}
                &
                \begin{itemize}
                    \item \textit{Hide Delegate}
                    \item \textit{Extract Method} i \textit{Move Method}
                \end{itemize}
                \\

            \end{tabular}
        \end{center}
    \end{table}
    \begin{table}[H]
        \begin{center}
            \begin{tabular}{ p{.35\linewidth} p{.65\linewidth}}

                \cmidrule(r){1-1}\cmidrule(l){2-2}
                \textbf{Pojemnik na dane}
                \begin{itemize}
                    \item Klasa jedynie przechowuje dane i nie posiada użytecznych metod.
                \end{itemize}
                &
                \begin{itemize}
                    \item \textit{Extract Method} i \textit{Move Method}
                    \item \textit{Inline Class}
                \end{itemize}
                \\

                \cmidrule(r){1-1}\cmidrule(l){2-2}
                \textbf{Zbitka danych}
                \begin{itemize}
                    \item Zbiór danych zawsze występujących wspólnie
                \end{itemize}
                &
                \begin{itemize}
                    \item \textit{Extract Class, Introduce Parameter Object, Preserve Whole Class}
                    \item podobnie jak z Long Parameter List
                \end{itemize}
                \\

                \cmidrule(r){1-1}\cmidrule(l){2-2}
                \textbf{Odrzucony spadek}
                \begin{itemize}
                    \item Podklasa nie wykorzystuje odziedziczonych metod i pól
                \end{itemize}
                &
                \begin{itemize}
                    \item \textit{Push Down Member}
                    \item \textit{Replace Inheritance with Delegation}
                \end{itemize}
                \\

                \cmidrule(r){1-1}\cmidrule(l){2-2}
                \textbf{Niewłaściwa hermetyzacja}
                \begin{itemize}
                    \item Klasa bezpośrednio odwołuje się do składowych wewnętrznych innej klasy.
                \end{itemize}
                &
                \begin{itemize}
                    \item \textit{Move Member}
                    \item \textit{Change Bidirectional References to Unidirectional}
                    \item \textit{Replace Inheritance with Delegation}
                \end{itemize}
                \\

                \cmidrule(r){1-1}\cmidrule(l){2-2}
                \textbf{Bezużyteczna klasa}
                \begin{itemize}
                    \item Klasa nie posiada żadnej lub ograniczoną funkcjonalność.
                \end{itemize}
                &
                \begin{itemize}
                    \item \textit{Move Member, Inline Class}
                    \item \textit{Collapse Hierarchy}
                \end{itemize}
                \\

                \cmidrule(r){1-1}\cmidrule(l){2-2}
                \textbf{Zazdrość o funkcję}
                \begin{itemize}
                    \item Metoda w klasie częściej korzysta z metod w obcych klasach
                    \item Niska spójność klasy
                \end{itemize}
                &
                \begin{itemize}
                    \item \textit{Move Method}
                    \item wzorzec projektowy Visitor
                \end{itemize}
                \\

                \cmidrule(r){1-1}\cmidrule(l){2-2}
                \textbf{Równoległe hierarchie dziedziczenia}
                \begin{itemize}
                    \item Utworzenie podklasy powoduje konieczność utworzenia odpowiadającej jej podklasy
                    w innej hierarchii dziedziczenia.
                \end{itemize}
                &
                \begin{itemize}
                    \item \textit{Move Method, Move Field, Extract Interface}
                    \item \textit{Inline Class}
                \end{itemize}
                \\

                \cmidrule(r){1-1}\cmidrule(l){2-2}
                \textbf{Pośrednik}
                \begin{itemize}
                    \item Klasa deleguje większość funkcjonalności do innych klas
                \end{itemize}
                &
                \begin{itemize}
                    \item \textit{Remove Middle Man, Inline Method}
                    \item \textit{Replace Delegation with Inheritance}
                \end{itemize}
                \\

                \cmidrule(r){1-1}\cmidrule(l){2-2}
                \textbf{Zmiany z wielu przyczyn}
                \begin{itemize}
                    \item Klasa jest wielokrotnie modyfikowana w różnych celach
                \end{itemize}
                &
                \begin{itemize}
                    \item \textit{Extract Class} z elementów zmieniających się z jednego powodu
                \end{itemize}
                \\


                \cmidrule(r){1-1}\cmidrule(l){2-2}
                \textbf{Odpryskowa modyfikacja}
                \begin{itemize}
                    \item Zmiana w jednym miejscu powodu konieczność modyfikacji w innych
                \end{itemize}
                &
                \begin{itemize}
                    \item \textit{Extract Class, Move Method, Move Field}
                    \item \textit{Inline Class}
                \end{itemize}
                \\

                \cmidrule(r){1-1}\cmidrule(l){2-2}
                \textbf{Spekulacyjne uogólnienie}
                \begin{itemize}
                    \item Klasa jest zaprojektowana pod kątem potencjalnej fnkcjonalności do zaimplementowania w przyszłości
                \end{itemize}
                &
                \begin{itemize}
                    \item \textit{Collapse Hierarchy}
                    \item \textit{Inline Class, Remove Parameter}
                    \item \textit{Rename Method}
                \end{itemize}
                \\
            \end{tabular}
        \end{center}
    \end{table}

    \subsection{Ewolucja oprogramowania}
    \textbf{Ewolucja oprogramowania} - \textbf{proces zmian} zachodzących w oprogramowaniu
    w czasie jego życia.

    \textbf{Pielęgnacja oprogramowania} - \textbf{czynności modyfikujące} program
    po jego dostarczeniu i wdrożeniu. Cele:
    \begin{itemize}
        \item poprawa błędów
        \item poprawa wydajności lub innych atrybutów programu
        \item adaptacja produktu do zmian w środowisku operacyjnym
    \end{itemize}


    \begin{table}[H]
        \begin{center}
            \begin{tabular}{ p{2cm} | p{4cm} | p{4cm} | p{4cm}}
                & \textbf{Program typu E} & \textbf{Program typu P} & \textbf{Program typu S}\\
                & osadzony w rz\textbf{E}czywistości & rozwiązujący \textbf{P}roblem & oparty na \textbf{S}pecyfikacji\\
                \toprule
                \textbf{założenia} & system funkcjonuje w rzeczywistym świecie
                & system w przybliżeniu odtwarza rzeczywistość
                & dostępna jest pełna specyfikacja systemu
                \\

                \cmidrule(r){2-2}\cmidrule(rl){3-3}\cmidrule(l){4-4}

                \textbf{kryterium jakości} & subiektywna ocena użytkownika
                & akceptowalne rozwiązanie problemu
                & zgodność ze specyfikacją
                \\

                \cmidrule(r){2-2}\cmidrule(rl){3-3}\cmidrule(l){4-4}
                \textbf{ewolucja} & nieunikniona, program i jego środowisko nieustannie oddziałują na siebie
                &
                prawdopodobna – poprawa programu, ewolucja środowiska
                &
                brak (modyfikacja nowy problem nowy program)
                \\
            \end{tabular}
        \end{center}
    \end{table}

    \subsubsection{Prawa Lehmana}
    Dotyczą systemów typu E, niesprawdzone w typach S i P, raczej obserwacje/hipotezy.

    \begin{table}[H]
        \begin{center}
            \begin{tabular}{p{.4\linewidth} p{.6\linewidth}}
                \textbf{Prawo nieustannej zmiany}
                &
                Program stosowany w
                rzeczywistym środowisku musi być stale do niego
                adaptowany. W przeciwnym przypadku będzie
                stopniowo coraz mniej użyteczny.
                \\

                \cmidrule(r){1-1}\cmidrule(l){2-2}

                \textbf{Prawo wzrastającej złożoności}
                &
                Wraz z ewolucją
                oprogramowania jego struktura staje się coraz
                bardziej złożona. Pielęgnacja i upraszczanie struktury
                wymagają dodatkowych nakładów.
                \\

                \cmidrule(r){1-1}\cmidrule(l){2-2}


                \textbf{Prawo samoregulacji}
                &
                Ewolucja oprogramowania jest
                samoregulującym procesem o rozkładzie atrybutów
                procesu i produktu bliskim rozkładowi normalnemu.
                Podstawowe atrybuty procesu ewolucji pozostają stałe
                dla każdego wydania.
                \\

                \cmidrule(r){1-1}\cmidrule(l){2-2}


                \textbf{Prawo organizacyjnej stabilności}
                &
                Tempo rozwoju
                oprogramowania w całym jego cyklu życia jest stałe,
                bez względu na dostępność zasobów, jeżeli podczas
                jego pielęgnacji nie wykorzystano właściwie informacji
                zwrotnej.
                \\

                \cmidrule(r){1-1}\cmidrule(l){2-2}



                \textbf{Prawo zachowania przyzwyczajeń}
                &
                W dłuższej
                perspektywie, rozmiar kolejnych wydań systemu
                jest statystycznie niezmienny. Przyswojenie nowych
                funkcji systemu wymaga czasu.
                \\

                \cmidrule(r){1-1}\cmidrule(l){2-2}


                \textbf{Prawo ciągłego wzrostu}
                &
                Funkcjonalność
                oprogramowania musi rosnąć, aby satysfakcja
                odbiorców systemu pozostała na tym samym
                poziomie.
                \\

                \cmidrule(r){1-1}\cmidrule(l){2-2}


                \textbf{Prawo spadku jakości}
                &
                Jakość oprogramowania spada,
                o ile nie jest ono dostosowywane do zmian
                zachodzących w swoim środowisku operacyjnym.
                \\

                \cmidrule(r){1-1}\cmidrule(l){2-2}


                \textbf{Prawo przyrostowego rozwoju}
                &
                Procesy ewolucyjne
                oprogramowania są wielopoziomowe, posiadają naturę
                iteracyjną i wymagają informacji z wielu punktów
                widzenia. Wykorzystanie informacji zwrotnej pozwala
                osiągnąć istotną poprawę procesu ewolucji.
                \\
            \end{tabular}
        \end{center}
    \end{table}

    \textbf{Wnioski z praw Lehmana}
    \begin{itemize}
        \item Oprogramowanie, aby pozostało użyteczne, musi ewoluować;
        \item Jakość oprogramowania (zdolność do ewolucji) pogarsza się z upływem czasu
        \item Rosnąca złożoność oprogramowania w pewnym momencie znacznie utrudnia dalszy rozwój systemu
        \item Tempo rozwoju oprogramowania jest w najlepszym przypadku stałe i nie zależy od
        sposobu zarządzania
    \end{itemize}

    \subsubsection{Pielęgnacja oprogramowania}
    Cztery rodzaje aktywności:
    \begin{itemize}
        \item \textbf{pielęgnacji doskonalącej} (ok. 50\%) - implementacja nowych wymagań
        funkcjonalnych;
        \item \textbf{pielęgnacji adaptacyjnej} (ok. 25\%) -  dostosowywanie do zmian zachodzących w środowisku;
        \item \textbf{pielęgnacji naprawcze} (ok. 20\%) - usuwanie błędów
        \item \textbf{prewencyjnej} (ok. 5\%) - restrukturyzacja wewnętrzna.
    \end{itemize}

    \textbf{Model kosztowy Boehma} - $AME = 1.0 * ACT * SDT$; AME - roczna pracochłonność związana
    z pielęgnacją [PM], ACT - względna liczba zmian [\%], SDT - pracochłonność rozwoju oprogramowania [PM].


    \begin{table}[H]
        \begin{center}
            \begin{tabular}{p{.5\linewidth} p{.5\linewidth}}
                \textbf{Czynniki wpływające na koszt pielęgnacji} & \textbf{Czynniki nie wpływające na koszt pilęgnacji}\\

                \begin{itemize}
                    \item  Dziedzina zastosowań systemu;
                    \item Stabilność personelu pielęgnacyjnego;
                    \item Czas życia oprogramowania;
                    \item Stabilność sprzętu;
                    \item Jakość kodu i dokumentacji.
                \end{itemize}
                &
                \begin{itemize}
                    \item Metoda zarządzania przedsięwzięciem;
                    \item Dostępność zasobów.
                \end{itemize}
            \end{tabular}
        \end{center}
    \end{table}

    \subsection{Zarządzanie konfiguracją}

    \begin{table}[H]
        \begin{center}
            \begin{tabular}{ p{.5\linewidth} p{.5\linewidth} }
                \textbf{Problemy} & \textbf{Procedury} \\
                \begin{itemize}
                    \item różnorodność artefaktów
                    \item równoległa praca nad fragmentami kodu
                    \item wiele wersji artefaktów
                    \item wersje artefaktów vs wersje produktu
                    \item analizowanie historii
                    \item praca nad nową wersją systemu i poprawianie starej
                \end{itemize}
                &
                \begin{itemize}
                    \item kodowanie;
                    \item wydawanie nowej wersji;
                    \item poprawianie defektów;
                    \item łączenia różnych zmian.
                \end{itemize}
                \\
            \end{tabular}
        \end{center}
    \end{table}

    \textbf{Narzędzia}
    \begin{itemize}
        \item CVS
        \item Subversion
        \item ClearCase
        \item Git
    \end{itemize}


\end{document}